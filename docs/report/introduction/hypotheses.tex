Although a considerable amount of work on breast cancer detection and diagnosis has been done in the institution, this project will be the first approximation to using convolutional networks for efficiently detecting breast cancer. Convolutional networks are widely used for object recognition tasks and have shown very good results~\cite{Russakovsky2015, Taigman2014, Dieleman2015}. They have a big research community and have become one of the preferred methods for image classification tasks.

Due to the exploratory nature of this work we are uncertain of the results that will be obtained. Nevertheless, we have a well established idea of what to expect. Our hypothesis is that applying convolutional networks to mammographic images will produce similar results to those obtained using more traditional computer vision techniques with less hassle. Additionally, we expect that a simple convolutional network will fail to obtain competitive results; we will need a convolutional network apt for image segmentation with well fitted hyperparameters. Furthermore, we believe that implementing convolutional networks for this domain will be moderately easy as other groups have already done it (Sec.~\ref{sec:BreastCancerConvNets}) and plenty of software is available.

\subsection{Research Questions}
Some of the questions which will be answered in this work are:
\begin{itemize} 
	%\item Can we improve the results reported by other groups using convolutional networks? Is training a convolutional network on mammographic images better than computing numeric features from the mammograms and training a simpler classifier?
	\item Are convolutional networks sufficiently powerful to perform breast cancer lesion segmentation as an end-to-end task? Is scarce data an obstacle for learning?
	\item Is deep learning feasible with the resources we have? Is our data and computational power sufficient? Is there any advantage to use GPU acceleration?
	\item Can we simplify the pipeline for breast cancer detection? Can preprocessing be replaced by more layers on the same convolutional network? Could we use the networks trained for image segmentation to perform detection or diagnosis?
	\item What are the best parameters for our convolutional networks (number of layers, number of units, kernel sizes, regularization, activation functions, etc)? Is there a big improvement on refining the network and tuning parameters?
%	\item Should we train a convolutional network for each type of breast lesion or could we use a single one with multiple outputs?
	\item What are the advantages of using a deep versus a shallow convolutional network? 
%	\item Could we use a convolutional network trained on a different database (such as the ImageNet database) to obtain features for mammographic images and use these features for classification?
	\item Are convolutional networks a good option for future research?
\end{itemize}
