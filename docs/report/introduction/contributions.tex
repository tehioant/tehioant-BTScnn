This thesis is the first documented use of convolutional networks for breast cancer lesion segmentation. It shows the viability of using modern machine learning techniques to simplify medical image processing pipelines and hints towards the viability of training more flexible models as larger data sets become available. Software, from preprocessing to evaluation tools, as well as some trained models are available online and have already been used by other groups working on similar tasks. Furthermore, a copy of the mammographic database and preprocessing tools remain available for use inside the institution. Lastly, this work supported the development of several deep learning projects by contributing theoretical and technichal knowledge.
\begin{comment}
 Describe the exact contributions of this thesis

 software to transform a new database into a segementaton problem
 software tools for evaluation.
 first result in segementation
 convolutional networks as a concept
 the trained model
 the start of deep learning on campus.

    What is new, different, better, significant?
    Why is the world a better place because of what you've done?
    What have you contributed to the field of research?
    What is now known/possible/better because of your thesis?
\end{comment}
